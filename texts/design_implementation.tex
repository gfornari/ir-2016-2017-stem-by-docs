\section{Software design and implementation}
    \subsection{Technologies}
        We chose to develop our project using Java. This programming language has many open source libraries and large active communities with millions of users. Moreover, programs written in Java are portable and therefore platform-independent.\par

        In order to make the management of our Java-based project easier, we chose to use Maven\footnotemark{}. This tool provides a uniform build system and guidelines for best practices development. Moreover, Maven is exceptionally good at managing external dependencies.\par

        \footnotetext{\url{https://maven.apache.org/}}
    \subsection{Design choices}
        We decided to respect a very important principle of software engineering: the design should be specific to the problem that you are facing, but also general enough for future requirements. If in the future someone will decide to use and extend the software that we developed, we want to avoid the need for a redesign.\par
        Breaking code up into modules helps to organize large code bases and makes programs easier to understand. But above all, modules are helpful for creating libraries that can be imported and used in different applications that share some functionality. We created many standalone modules that can be used and extended even outside the context of this project, as shown in Fig. \ref{stemby-package}.\par
        \begin{figure}
			\centering
			\begin{tikzpicture}
    % com.stemby
    \node at (0.50\textwidth, 0.42\textwidth) {\code{com.stemby}};
    \draw[fill=white] (0, 0) rectangle ++(1.0\textwidth, 0.40\textwidth);

    % commons
    \node at (0.15\textwidth, 0.38\textwidth) {\code{commons}};
    \draw[fill=blue1] (0.02\textwidth, 0.02\textwidth) rectangle ++(0.27\textwidth, 0.34\textwidth);

    % ir
    \node at (0.66\textwidth, 0.38\textwidth) {\code{ir}};
    \draw[fill=blue1] (0.35\textwidth, 0.02\textwidth) rectangle ++(0.63\textwidth, 0.34\textwidth);

    %commons.algorithms
    \draw[fill=blue2] (0.04\textwidth, 0.04\textwidth) rectangle ++(0.23\textwidth, 0.14\textwidth)
        node[pos=.5, text width=0.26\textwidth, align=center] {\code{algorithms}};
    
    % commons.util
    \draw[fill=blue2] (0.04\textwidth, 0.20\textwidth) rectangle ++(0.23\textwidth, 0.14\textwidth)
        node[pos=.5, text width=0.26\textwidth, align=center] {\code{util}};
    
    % ir.io
    \draw[fill=blue2] (0.37\textwidth, 0.27\textwidth) rectangle ++(0.28\textwidth, 0.07\textwidth)
        node[pos=.5, text width=0.26\textwidth, align=center] {\code{io}};
    
    % ir.util
    \draw[fill=blue2] (0.68\textwidth, 0.27\textwidth) rectangle ++(0.28\textwidth, 0.07\textwidth)
        node[pos=.5, text width=0.26\textwidth, align=center] {\code{util}};

    % ir.algorithms
    \node at (0.67\textwidth, 0.24\textwidth) {\code{algorithms}};
    \draw[fill=blue2] (0.37\textwidth, 0.04\textwidth) rectangle ++(0.59\textwidth, 0.18\textwidth);

    % ir.algorithms.stemming
    \node at (0.67\textwidth, 0.19\textwidth) {\code{stemming}};
    \draw[fill=blue3] (0.39\textwidth, 0.06\textwidth) rectangle ++(0.55\textwidth, 0.11\textwidth);

    % ir.algorithms.stemming.sns
    \draw[fill=blue4] (0.41\textwidth, 0.08\textwidth) rectangle ++(0.51\textwidth, 0.07\textwidth)
        node[pos=.5, text width=0.26\textwidth, align=center] {\code{sns}};
\end{tikzpicture}
            \caption{Structure of the package \code{com.stemby}}
			\label{stemby-package}
		\end{figure}
    \subsection{Analysis of the main problems}