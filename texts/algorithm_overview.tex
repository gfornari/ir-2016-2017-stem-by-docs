\section{Algorithm overview}
% descrizione sns
    In this section we briefly highligth the algorithm's logic; for a in-depth description, refer to the original paper.
 
    The goal of SNS is to group morphologically related words in different clusters, using the frequencies of every word in each documents. The algorithm is divided into three main parts:

    \begin{enumerate}
        \item computing co-occurence strength of words pairs, in order to create a graph where co-occuring words (if they meet certain conditions on prefixes and suffixes) are connected;
        \item re-calculating co-occurence strength considering neighbors in the graph;
        \item grouping words considering the new co-occurence computed.
    \end{enumerate}

    In order to achieve all of these steps, we only need the occurencies of each words on all documents, in order to compute the initial co-occurence strength. The other step of the algorithm are based on computation of these data and operation on each terms. 