\section{Algorithm overview}
% descrizione sns
    In this section we briefly highlight the algorithm's logic; for an in-depth description, refer to the original paper\cite{sns}.
 
    The goal of SNS is to group morphologically related words in different clusters, using the frequencies of every word in each document. The stemmer does not make use of language-specific rules, but only information about occurrences of the word: it is language-independent.
    
The algorithm consists of three main parts:

    \begin{enumerate}
        \item computing co-occurrence strength of words pairs, in order to create a graph where co-occurring words are connected (if they meet certain conditions on prefixes and suffixes);
        \item re-calculating co-occurrence strength considering neighbours in the graph;
        \item grouping words considering the new co-occurrence computed.
    \end{enumerate}

    In order to achieve all of these steps, we only need the occurrences of each word on all documents, which are used to compute the initial co-occurrence strength. The other steps of the algorithm are based on the computation of these data. In particular, the re-calculating of the co-occurrence strength is achieved using this formula:
    
    \begin{equation}
		RCO(a,b) = CO(a,b) + \sum_{c \in N_{a,b}}min(CO(a,c), CO(c, b)) * 0.5    
    \end{equation}
    The authors did not specify what the parameter ``0.5'' stands for. They did not even explained why it is the best choice.
    
Another ambiguity in the paper is the choice of the stem itself. The output of the operations described is a collection of clusters grouping words with the same stem. However, the authors did not explain how to choose the stems from the clusters. We kept as stem the longest common prefix of the words of the same cluster. A consequence of this solution is that the same stem can be chosen for different clusters.