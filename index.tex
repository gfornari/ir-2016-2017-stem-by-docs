\documentclass[10pt,a4paper]{article}
\usepackage[utf8]{inputenc}
\usepackage[english]{babel}
\usepackage{amsmath}
\usepackage{amsfonts}
\usepackage{amssymb}

\usepackage{enumitem}

\author{Stem-by\\Riccardo Bucco, Giacomo Fornari, Alessandro Tezza}
\title{Reproducibility of SNS \\Information Retrieval - University of Padua}

\begin{document}
\maketitle
\section{Introduction}

\section{Problem Analysis}
\subsection{Feasibility study}
% Dettaglio del paper: preciso, pseudocodice
% Conoscenze: operazioni sui grafi, strutture dati sui grafi. Nessuna conoscenza oscura al team
The assigned paper has precise technical details. In fact, every operation is described accurately, using pseudocode to explain the algorithm. With such level of details, we thought that the implementation phase wouldn't be the most difficult part. 	
\paragraph{Difficulties}
% Gestione efficientemente grande quantità di dati, ci stanno in RAM? Struttura dati adeguata per garantire efficienza in termini di spazio e tempo
% parametro non motivato: eseguiremo test per vedere migliore efficacia
There are some parts in which the paper isn't totally clear:
\begin{itemize}
\item there is not a specification about how choose the stem for a cluster;
\item in Section 3.2, the first condition for the connection of two words with an edge are not so clear: 
\begin{quotation}
``A pair of words w1 and w2 is connected by an edge (w1, w2) if CO(w1 , w2 )\textgreater 0 and they have a common prefix of a given length (we chose 3) along with the suffixes (after truncating the longest common prefix) which are the residues (the ends after removal of longest common prefix) of more than one pair of co-occurring words with long common prefix (length larger than 5 which is the second static parameter, say, l2 ).''
\end{quotation}

\item in section 3.2, in formula (3), the writers didn't specify how it was found the parameter ``0.5'', and why it is the better choice.
\end{itemize}
\paragraph{Estimated time}
% Poco tempo per raggiungere un primo prototipo funzionante, almeno un mese per evaluation e migliorie efficienza. Compatibilmente a impegni del team, due settimane circa per progettazione-codifica, un mese per evaluation, migliorare prototipo

\subsection{Goals}
\section{Planning}
Considering the requirements of the projects and the importance of the evaluation of the system, we decided to split the work in two main periods.
In the first period we dedicated our efforts on setting up the environment (i.e., Terrier) and on the implementation of a first prototype of the stemmer. With this prototype we didn't take account of the efficiency. The only goal was that the stemmer had to do its work without errors. The second period was dedicated on the evaluation of the stemming algorithm and on the tuning of our implementation, in order to achieve better efficiency. We planned to dedicate more effort to the second period for three main reasons:
\begin{enumerate}
\item we consider the evaluation the most important part on this project;
\item thanks to the accurate implementation details on the paper, we have estimated more time for the evaluation and the efficiency tuning than the implementation itself;
\item we can repair problems occurred in the implementation period without running out of time.
\end{enumerate}

Furthermore, we decided to start the writing of this relation at the beginning of the project. In this way, we keeping track of every internal choices of the team in order to have a written report of our work, so as to avoid misunderstandings.
\subsection{Deadlines}
On the basis of the considerations made, we set the following internal deadline:
\begin{enumerate}[label=\textbf{D.\arabic*}]
\item \label{firstDeadline} 10th December 2016: end of the first period. We wanted to have a complete and working version of the stemmer, without considering efficiency matters.
\item \label{secondDeadline} \textit{date not agreed}: during this period we focused on the evaluation of the system and on the tuning of the efficiency of the system;
\item \label{thirdDeadline} not yet agreed;
\end{enumerate}
\subsection{Role Assignment}
We agreed in the rotation of our roles, so that each one of us can work with all the different aspects of the project. 

The assignment of roles in the different periods was:

\begin{center}
    \begin{tabular}{| l | l | l | l | l |}
    \hline
    & Until \ref{firstDeadline} & Until \ref{secondDeadline} & Until \ref{thirdDeadline} \\ \hline
	Riccardo Bucco & Setting the environment. & Evaluation/Writing the report. & \\ \hline
	Giacomo Fornari & Implementation & Evaluation. & \\ \hline
	Alessandro Tezza & Writing the report. & Efficiency tuning. &  \\ \hline
    \end{tabular}
\end{center}


\subsection{Adopted Technologies}
We considered two language for the development of the stemmer: Java and Python. Both of them are largely used, with active communities behind them. Java is the recommended language and it is the language used in Terrier. Python is largely used in numerical computation. For this reason, it is particulary efficient in operations on matrix, useful for our purpose. Other advantages of Python are his simple syntax and his high-level structures, that can result in shorter develop time. However, considering that none of us had experience with Python, this reduced time would be replaced by our study of the language. 
Generally, Java has better performance, because Python is an interpeted language, as it is told by Python developer themselves\footnote{Comparing Python to Other Languages: https://www.python.org/doc/essays/comparisons/}. Principally for this latter reason and considering that the stemmer had to work with an high amount of documents, we choosed to develop it using Java.

\subsection{Implementation choices}
The main difficulties in term of implementation of the stemmer was finding a data structure that allows to storage the different graphs used by the algorithm. For our first prototype, we used a simple matrix with every word in the rows and every word in the columns. The result was an highly inefficent data storage, considering that we have to store N*N values, with N the number of words. Furthermore, the resulting matrix was a sparse one, i.e., there was a largest part of the matrix with the value 0.

\section{Coding}
\subsection{Difficulties}
We had some difficulties in the understanding of the computation of RCO. At the first time, in fact, we didn't understand that the RCO had to be computed on all the words that had a cooccorence: instead, we considered all the words grouped in the same graph using the previous clustering. In fact, the paper state:
\begin{quotation}
``Here we consider the word pairs (a, b) for which CO(a, b) > 0.''
\end{quotation}
For this reason, our first implementation was wrong: we had then some delay to complete our first prototype before deadline \ref{firstDeadline}, and we started the second period after we planned.
\section{Evaluation}
\subsection{Efficiency}

\end{document}